\documentclass{article}
\usepackage[backend=bibtex, style=numeric]{biblatex}
\usepackage{amsmath}
\usepackage{graphicx}
\usepackage{wrapfig}
\usepackage{float}
%\usepackage{caption}
%\usepackage{subcaption}
\usepackage{subfigure}
\usepackage{capt-of}
\usepackage[utf8]{inputenc}
\usepackage{graphicx}
%\usepackage{subcaption}
\usepackage{bbold}
\usepackage{float}
\usepackage{wrapfig}
\usepackage{tablefootnote}
\usepackage{listings}
%\usepackage[backend=biber, style=numeric]{biblatex}
\usepackage{framed, color}
\usepackage[german]{babel}
%\usepackage{hyperref}
\usepackage{url}
\usepackage{epigraph}
%\usepackage{ngerman}
%\usepackage[numbers]{natbib}


\definecolor{shadecolor}{rgb}{1.0, 0.8, 0.3}



\renewcommand{\thesection}{\S\,\arabic{section}}


%author and title section
\title{Satzung des ACM Munich CS Student Club}
\date{Stand: 6. Dezember 2015}
%\institute{Technische Universität München  \\[5mm]
%{\tt\small frederikdiehl@gmail.com} }

\begin{document}
\newcommand{\zit}[2]{"#1"\footnote{#2}}
\newcommand{\einf}[2]{\center{"#1"}.\arabic{subsection}}
\maketitle

\begin{center}
Beschlossen auf der Gründungsversammlung am 6. Dezember 2015 in München.
%Eingetragen im Vereinsregister des Amstgerichtes München unter der Registriernummer VR XXXX am XXXX.
\end{center}



\section*{Präambel}
\label{sec:praeambel}
Der ACM Munich CS Student Club hat als Ziel, das Wissen über und Interesse an der Informatik zu fördern.
In diesem Sinne gibt sich der ACM Munich CS Student Club folgende Satzung:

\section{Name, Sitz, Geschäftsjahr}
\label{sec:namesitz}
\begin{enumerate}
\item Der Verein führt den Namen ACM Munich CS Student Club.\\
Er soll in das Vereinsregister eingetragen werden und trägt dann den Zusatz \textit{e.V.}

\item Er hat seinen Sitz in München.

\item Das Geschäftsjahr ist das Kalenderjahr.
\end{enumerate}

\section{Ziele und Aufgaben des Vereins}
\label{sec:ziele}
\begin{enumerate}
\item Das Ziel des Vereins ist es, das studentische und nicht-studentische Interesse an und Wissen über die Informatik zu fördern.
\item Der Verein erreicht seine Ziele insbesondere durch
\begin{enumerate}
\item Veranstaltung von Präsentationen und Vorträgen
\item Veranstaltung von Workshops
\item Bieten einer Möglichkeit zur Zusammenarbeit und Austausch innerhalb des Vereins
\end{enumerate}
  
\end{enumerate}
  
\section{Gemeinnützigkeit} 
\label{sec:gemeinnuetzigkeit}  
\begin{enumerate}
\item Der Verein verfolgt ausschließlich und unmittelbar gemeinnützige Zwecke im Sinne des Abschnitts „Steuerbegünstigte Zwecke“ der Abgabenordnung. Der Verein ist selbstlos tätig; er verfolgt nicht in erster Linie eigenwirtschaftliche Zwecke.
\item Mittel des Vereins dürfen nur für die satzungsmäßigen Zwecke verwendet werden. Die Mitglieder erhalten in ihrer Eigenschaft als Mitglied keine Zuwendungen aus Mitteln des Vereins. Sie haben bei ihrem Ausscheiden keinerlei Ansprüche an das Vereinsvermögen. Keine Person darf durch Ausgaben, die den Zwecken des Vereins fremd sind oder durch unverhältnismäßig hohe Vergütungen begünstigt werden.
\end{enumerate}

\section{Mitgliedschaft}
\label{sec:mitgliedschaft}
\begin{enumerate}
\item Mitglieder können alle natürlichen und juristischen Personen werden, die die Ziele des Vereins unterstützen.
\item Der Mitgliedschaftsantrag wird durch Antrag in Textform an den Vorstand bekundet und vom Vorstand baldmöglichst entschieden.
\item Die Mitgliedschaft endet, abgesehen von Tod oder Auflösung, durch:
	\begin{enumerate}
		\item Austrittserklärung in Textform seitens des Mitglieds
		\item Streichung aus der Mitgliederliste. Die Streichung eines Mitglieds aus der  Mitgliederliste kann durch die Mitgliederversammlung erfolgen, wenn das Mitglied über mehr als drei Monate ohne Angabe von Gründen inaktiv ist und auf mindestens einen einen Monat vor der Mitgliederversammlung zugesandten Hinweis in Textform nicht reagiert.
		\item Die Streichung eines Mitglieds aus der Mitgliederliste erfolgt ebenso durch den Vorstand, wenn das Mitglied mit der Zahlung des Mitgliedsbeitrags mehr als einen Monat im Verzug ist und diesen trotz zweimaliger Mahnung, die in Textform zu erfolgen hat, nicht gezahlt hat.
		\item Ein Mitglied wird aus dem Verein ausgeschlossen, wenn es in erheblichem Maße die Ziele des Vereins verletzt. Über den Ausschluss entscheidet der Vorstand. Gegen einen Ausschluss kann innerhalb von vier Wochen schriftlich Widerspruch eingelegt werden. Über diesen Widerspruch entscheidet die nächste Mitgliederversammlung.
	\end{enumerate}


\end{enumerate}

\section{Mitgliedsbeitrag}
\label{sec:mitgliedsbeitrag}
\begin{enumerate}
\item Die Mitgliederversammlung erlässt eine Beitragsordnung, die die Höhe und Existenz der pro Semester zu zahlenden Beiträge regelt.
\item Die Beitragsordnung regelt ebenso die Höhe und Existenz einer Aufnahmegebühr welche einmalig bei Aufnahme in den Verein anfällt.
\end{enumerate}

\section{Organe des Vereins}
\label{sec:organe}
\begin{enumerate}

\item Die Organe des Vereins sind:
	\begin{enumerate}
		\item Mitgliederversammlung
		\item Vorstand
		\item Kassenprüfer
	\end{enumerate}
\end{enumerate}

\section{Mitgliederversammlung}
\label{sec:mitgliederversammlung}
\begin{enumerate}
	\item Oberstes Organ des Vereins ist die Mitgliederversammlung. Sie tagt zweimal im Jahr, jeweils zum Ende des Semesters. Versammlungsleiter ist in der Regel ein Vorstandsmitglied. Der Vorstand hat bei jeder ordentlichen Mitgliederversammlung einen Geschäftsbericht zu formulieren und ist nach Feststellung des Semesterabschlusses zu entlasten. Die Mitgliederversammlung entscheidet ebenso über eine Eintragung in das  Vereinsregister.
	\item Die Mitgliederversammlung stellt die Richtlinien für die Arbeit des Vereins auf und  entscheidet in Fragen von grundsätzlicher Bedeutung. Zu den Aufgaben der  ordentlichen Mitgliederversammlung gehören insbesondere:
	\begin{enumerate}
		\item Beschlussfassung über Änderungen der Satzung und Auflösung des Vereins
		\item Entgegennahme des Geschäftsberichtes des Vorstandes
		\item Beschlussfassung über den Semesterabschluss
		\item Beschlussfassung über die Entlastung des Vorstandes
		\item Wahl und Abwahl des Vorstands
		\item Wahl des Kassenprüfers
		\item Festlegung der Höhe der Mitgliedsbeiträge und der Aufnahmegebühr
	\end{enumerate}
	\item Zur Mitgliederversammlung wird von einem Vorstandsmitglied unter Angabe der  vorläufigen Tagesordnung mindestens zwei Wochen vorher eingeladen. Eingeladen
 wird per E-Mail, auf schriftlichen Antrag auch auf dem Postweg.
 \item Der Vorstand kann jederzeit eine außerordentliche Versammlung der Mitglieder  einberufen. Er muss sie einberufen, wenn ein Viertel der Mitglieder dies schriftlich  unter Angabe von Gründen verlangt. Sie muss längstens fünf Wochen nach Eingang  des Antrags auf schriftliche Berufung tagen.
 \item Die Mitgliederversammlung ist beschlussfähig, wenn sie ordnungsgemäß einberufen worden ist. Zur Beschlussfassung ist die einfache Stimmenmehrheit der erschienenen Mitglieder erforderlich. Stimmengleichheit bedeutet Ablehnung. Die Abstimmung erfolgt per Handzeichen, auf Verlangen eines anwesenden Mitglieds jedoch geheim.
 \item Über die Beschlüsse und, soweit zum Verständnis über deren Zustandekommen  erforderlich, auch über den wesentlichen Verlauf der Verhandlung, ist eine Niederschrift anzufertigen. Sie wird vom Versammlungsleiter und dem Protokollführer unterschrieben.
\end{enumerate}

\section{Vorstand}
\label{sec:vorstand}
\begin{enumerate}
	\item Der Vorstand besteht aus zwei Mitgliedern. Sie bilden den Vorstand im Sinne von \S\,26 BGB. Das Vorstandsamt endet mit der Mitgliedschaft im Verein. Die Vorstandsmitglieder sind ehrenamtlich tätig.
	\item Zur rechtsverbindlichen Vertretung genügt die Zeichnung eines Vorstandsmitgliedes. Im Innenverhältnis ist vereinbart, dass der Vertreter nach außen dasjenige Vorstandsmitglied ist, welches bereits länger im Amt ist.
	\item Die Amtszeit der Vorstandsmitglieder beträgt 1 Jahr, wobei bei jeder regulären Mitgliederversammlung nur ein Vorstandsmitglied gewählt wird, d.h. dass die Amtszeit der beiden Vorstandsmitglieder um ein halbes Jahr verschoben ist um die Kontinuität des Vereins zu gewährleisten. Bei Gründung wird ein Mitglied auf ein Jahr in den Vorstand berufen, ein anderes für ein Semester. Sie bleiben jeweils bis zur Bestellung des neuen Vorstandsmitgliedes im Amt. Scheidet ein Mitglied des Vorstands während der Amtsperiode aus, erfolgt eine Nachwahl für die restliche Amtsdauer durch den Vorstand. Eine vorzeitige Abwahl des Vorstands durch die Mitgliederversammlung aus wichtigem Grund ist jederzeit möglich.
	\item Die beiden Vorstandsmitglieder führen sämtliche Geschäfte des Vereins und vollziehen die Beschlüsse der Mitgliederversammlung.
\end{enumerate}

\section{Kassenprüfer}
\label{sec:kassenpruefer}
\begin{enumerate}
	\item Um die sachgerechte und wirtschaftliche Verwendung der Mittel des Vereins zu überprüfen, bestellt die Mitgliederversammlung im Wintersemester einen Kassenprüfer für das nächste Jahr. Er prüft auch den Jahresabschluss.
	\item Zur Wahrnehmung seiner Aufgaben kann er vom Vorstand alle erforderlichen Auskünfte mündlich und/oder schriftlich und die Einsicht in alle Unterlagen verlangen. Er erstattet jeder ordentlichen Mitgliederversammlung einen Bericht.
	\item Zum Kassenprüfer wird von der Mitgliederversammlung ein Mitglied des Vereins bestellt. Zur Wahrung der Objektivität des Kassenprüfers darf dieser kein Mitglied des Vorstandes sein.
\end{enumerate}

\section{Satzungsänderungen und Auflösung}
\label{sec:satzungsaenderung}
\begin{enumerate}
	\item Über Satzungsänderungen, die Änderung des Vereinszwecks und die Auflösung  entscheidet die Mitgliederversammlung. Vorschläge zu Satzungsänderungen, Zweckänderungen und zur Auflösung sind den stimmberechtigten Mitgliedern bis spätestens einen Monat vor der Sitzung der Mitgliederversammlung zuzuleiten. Für die Beschlussfassung ist eine Mehrheit von drei Vierteln der anwesenden Stimmberechtigten erforderlich.
	\item Änderungen oder Ergänzungen der Satzung, die von der zuständigen Registerbehörde oder vom Finanzamt vorgeschrieben werden, werden vom Vorstand umgesetzt und bedürfen keiner Beschlussfassung durch die Mitgliederversammlung. Sie sind den Mitgliedern spätestens mit der nächsten Einladung zur Mitgliederversammlung mitzuteilen.
	\item Bei Auflösung, bei Entziehung der Rechtsfähigkeit des Vereins oder bei Wegfall der steuerbegünstigten Zwecke fällt das gesamte Vermögen an die Fachschaft MPI der TU München, und zwar mit der Auflage, es entsprechend seinen bisherigen Zielen und Aufgaben ausschließlich und unmittelbar gemäß \ref{sec:ziele} zu verwenden.
\end{enumerate}

\vspace{2cm}
\hspace{2cm}Ort\hspace{7cm}Datum
\vspace{5cm}
\begin{center}
Unterschriften
\end{center}
\end{document}

%http://www.livius.org/people/mamertines/